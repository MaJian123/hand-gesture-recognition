\subsection{Classifieur histogramme radial}
\subsubsection{Présentation de la méthode}

Cette méthode de classification consiste à extraire comme vecteur caractéristique de la main un histogramme radial pondéré des pixels de la main autour d'un centre, pour ensuite classifier ce vecteur à l'aide de classifieurs Bayésiens, $K$ plus proches voisins ou réseau de neurone à une couche cachée. Cette technique avait été utilisée avec succès pour la reconnaissance de signes de la main avec une caméra Kinect\cite{matthewTang}, c'est pourquoi nous avons jugé intéressant de l'implémenter.

\paragraph{Calcul de l'histogramme radial}
Intuitivement, l'histogramme radial consiste à calculer un histogramme de projection des pixels de la main "autour" du centre de gravité de la main. Pour déterminer celui-ci, définissons tout d'abord $I$ l'image de la main segmentée et binarisée:

\[
I(x,y) = \left\{
  \begin{array}{l l}
    1 & \quad \text{si $(x,y)$ est un pixel de la main}\\
    0 & \quad \text{sinon}
  \end{array} \right.
\]

\begin{figure}[htb!]
\caption{Exemple d'une image de main segmentée et binarisée.}
\label{fig:segBinRadial}
\end{figure}

Nous déterminons ensuite une image $I'$ de décalage d'angles, qui associe, pour chaque droite passant par le centre de gravité $c = (x_c, y_c)$ de la main, un niveau de gris distinct. C'est en calculant l'histogramme des niveaux de gris de cette image que nous obtenons l'histogramme radial. Ainsi, nous définissons $I'$ de la façon suivante pour tout $(x,y)$ de la main:

\[
I'(x,y) = tan^{-1}(\frac{x - x_c}{y - y_c})
\]

\begin{figure}[htb!]
\caption{Image de décalages d'angles obtenue à partir de (\autoref{fig:segBinRadial})}
\label{fig:decalageAngles}
\end{figure}

Nous calculons alors l'histogramme des niveaux de gris de cette image, et nous obtenons alors des histogrammes donnant des pics distincts pour chaque doigt de la main (\autoref{fig:histoRadialGravite}).

\begin{figure}[htb!]
\caption{Histogramme radial obtenu à partir de (\autoref{fig:segBinRadial})}
\label{fig:histoRadialGravite}
\end{figure}

Cependant, le centre de gravité a tendance à être positionné là où le pouce et l'auriculaire ont le même niveau de gris dans l'image de décalages d'angles (\autoref{fig:segBinRadial}). C'est pourquoi nous utilisons à la place du centre de gravité de la main le centre de la paume de la main, que nous calculons en supprimant les doigts de l'image binarisée par érosion avec un élément structurant suffisamment grand pour supprimer entièrement les doigts de la main. Celui-ci étant placé plus bas sur la main, celui-ci produit des pics plus distincts pour chaque doigt dans l'histogramme radial (\autoref{fig:histoRadialPaume}).

\begin{figure}[htb!]
\caption{Image segmentée et binarisée d'une main, avec en vert le centre de gravité de la main et en rouge le centre de gravité de la main.}
\label{fig:segBinRadialCentres}
\end{figure}

\begin{figure}[htb!]
\caption{Histogramme radial avec centre de la paume obtenu à partir de (\autoref{fig:segBinRadial})}
\label{fig:histoRadialPaume}
\end{figure}

L'histogramme radial ainsi obtenu constitue le vecteur caractéristique de la main. Nous avons utilisé plusieurs méthodes pour classifier ce vecteur caractéristique:

\begin{itemize}
\item Classifieur Bayésien simple, sans spécificité particulière. Nous avons pour cela utilisé la classe CvNormalBayesClassifier du module machine learning d'OpenCV.
\item $K$ plus proches voisins, encore une fois sans spécificités. Nous avons pour cela utilisé la classe CvKNearestClassifier du module machine learning d'OpenCV.
\item Réseau de neurones artificiels, avec une topologie à $3$ couche:
\end{itemize}
